\documentclass[14pt,a4paper]{article}

\usepackage{textgreek}
\usepackage[utf8x]{inputenc}
\usepackage{graphicx}
\usepackage[export]{adjustbox}
\usepackage[a4paper, portrait, margin=1in]{geometry}
\usepackage[ukrainian]{babel}
\usepackage{cmap}
\usepackage{etoolbox}
\usepackage{caption}
\usepackage{booktabs}
\usepackage{listings}
\usepackage{pgfplots}
\usepackage{xcolor} 
\usepackage{titlesec}
\usepackage{setspace}
\usepackage{fancyhdr} 
\usepackage{amsmath} 
\usepackage{amsthm}
\usepackage{hyperref}
\usepackage{amsmath} 
\usepackage{bm} 
\usepackage[square,sort,comma,numbers,super]{natbib}
\usepackage{caption}
\usepackage{float}

\graphicspath{ {./Images/} }

\title{\Huge \textbf{Інструкція для користувача  до системи AtOM. Access to Memory} }
\date{}

\begin{document}

\begin{titlepage}
    \pagecolor{blue} 
    \color{white}
    \maketitle
    \thispagestyle{empty}
    
	\begin{center}
	\includegraphics[max width=1.5\textwidth]{Images/logo_bl.png}
	\end{center}    
    
    \vspace*{8cm}
    \center \textbf{Львівський національний університет імені Івана Франка}
    \center \textbf{Львів 2024}

\end{titlepage}

\pagecolor{white}
\color{black}

\newpage

\pagenumbering{Roman}
\tableofcontents
\newpage
\pagenumbering{arabic}
\section{Вступ}
Access to Memory (AtoM) — це програмне забезпечення для управління архівними описами, розроблене відповідно до міжнародних стандартів, таких як ISAD(G) та EAD. Цей посібник пояснює основні операції, які можна виконати в AtoM.

\section{Робота з архівними описами}

AtoM (Access to Memory) підтримує створення, редагування та управління архівними описами відповідно до міжнародних стандартів ISAD(G). Цей розділ пояснює, як виконувати основні операції з архівними описами.

\subsection{Створення нового архівного опису}
Щоб створити новий архівний опис, дотримуйтесь таких кроків:
\begin{enumerate}
	\item Увійдіть у систему як користувач із правами на створення записів.
	\item Перейдіть до меню \textbf{Add > Archival description}.
	\item Заповніть основні поля:
	\begin{itemize}
		\item \textbf{Назва} — обов'язкове поле для введення основного заголовку запису.
		\item \textbf{Рівень опису} — вкажіть, чи це фонд, серія, файл або одиниця зберігання.
		\item \textbf{Дати} — задайте часовий діапазон, пов'язаний із записом.
	\end{itemize}
	\item Натисніть \textbf{Save}, щоб зберегти опис.
\end{enumerate}

\subsection{Редагування архівного опису}
Щоб внести зміни до вже існуючого запису:
\begin{enumerate}
	\item Використовуйте функцію пошуку для знаходження запису, який потрібно редагувати.
	\item Перейдіть до сторінки опису та натисніть кнопку \textbf{Edit}.
	\item Внесіть необхідні зміни в текст або додайте нові метадані.
	\item Натисніть \textbf{Save}, щоб зберегти оновлений запис.
\end{enumerate}

\subsection{Видалення архівного опису}
Видалення архівного опису можливе тільки для користувачів із правами адміністратора:
\begin{enumerate}
	\item Знайдіть запис, який потрібно видалити, за допомогою пошуку.
	\item Відкрийте сторінку запису та натисніть кнопку \textbf{Delete}.
	\item Підтвердіть видалення у діалоговому вікні.
\end{enumerate}

\subsection{Додавання цифрових об'єктів}
Цифрові файли, такі як зображення, PDF-документи або аудіо/відео файли, можуть бути прикріплені до архівних описів:
\begin{enumerate}
	\item Перейдіть до архівного запису, до якого потрібно додати цифровий об'єкт.
	\item Відкрийте вкладку \textbf{Digital objects}.
	\item Натисніть кнопку \textbf{Upload} для завантаження файлу.
	\item Заповніть додаткові поля метаданих для цифрового об'єкта (за потреби).
	\item Натисніть \textbf{Save}, щоб зберегти зміни.
\end{enumerate}


\section{Керування цифровими об'єктами}
\subsection{Додавання цифрових об'єктів}
Цифрові об'єкти, такі як зображення чи PDF-документи, можна додати до архівного опису:
\begin{enumerate}
	\item Увійдіть у запис, до якого ви хочете додати об'єкт.
	\item Перейдіть на вкладку \textbf{Digital objects}.
	\item Завантажте файл і натисніть \textbf{Save}.
\end{enumerate}

\section{Пошук та навігація}
\subsection{Пошук}
AtoM підтримує простий та розширений пошук:
\begin{itemize}
	\item Для простого пошуку використовуйте поле \textbf{Search} у верхній частині сторінки.
	\item Для розширеного пошуку оберіть \textbf{Advanced search} і задайте додаткові фільтри.
\end{itemize}

\subsection{Навігація по ієрархії}
Користуйтеся деревоподібною структурою для перегляду пов'язаних записів:
\begin{itemize}
	\item Натисніть на значок \textbf{Hierarchy} на сторінці опису.
	\item Оберіть потрібний рівень для перегляду підрозділів.
\end{itemize}

\section{Управління користувачами}
\subsection{Створення користувачів}
Адміністратор може створювати нових користувачів:
\begin{enumerate}
	\item Перейдіть до меню \textbf{Admin > Users > Add user}.
	\item Заповніть інформацію та встановіть рівень доступу.
	\item Натисніть \textbf{Save}.
\end{enumerate}

\section{Експорт даних}
Для експорту архівних описів:
\begin{enumerate}
	\item Увійдіть у меню \textbf{Export}.
	\item Оберіть формат експорту (EAD, CSV тощо).
	\item Натисніть \textbf{Generate}.
\end{enumerate}
\end{document}
