\documentclass[14pt,a4paper]{article}

\usepackage{textgreek}
\usepackage[utf8x]{inputenc}
\usepackage{graphicx}
\usepackage[export]{adjustbox}
\usepackage[a4paper, portrait, margin=1in]{geometry}
\usepackage[ukrainian]{babel}
\usepackage{cmap}
\usepackage{etoolbox}
\usepackage{caption}
\usepackage{booktabs}
\usepackage{listings}
\usepackage{pgfplots}
\usepackage{xcolor} 
\usepackage{titlesec}
\usepackage{setspace}
\usepackage{fancyhdr} 
\usepackage{amsmath} 
\usepackage{amsthm}
\usepackage{hyperref}
\usepackage{amsmath} 
\usepackage{bm} 
\usepackage[square,sort,comma,numbers,super]{natbib}
\usepackage{caption}
\usepackage{float}

\graphicspath{ {./Images/} }

\title{\Huge \textbf{Інструкція для користувача  до системи AtOM. Access to Memory} }
\date{}

\begin{document}

\begin{titlepage}
    \pagecolor{blue} 
    \color{white}
    \maketitle
    \thispagestyle{empty}
    
	\begin{center}
	\includegraphics[max width=1.5\textwidth]{Images/logo_bl.png}
	\end{center}    
    
    \vspace*{8cm}
    \center \textbf{Львівський національний університет імені Івана Франка}
    \center \textbf{Львів 2024}

\end{titlepage}

\pagecolor{white}
\color{black}

\newpage

\pagenumbering{Roman}
\tableofcontents
\newpage
\listoffigures
\newpage
\listoftables
\newpage
\pagenumbering{arabic}

\section{Загальна інформація}
This is the introduction section.

\section{Технічні вимоги}

\section{Встановлення системи AtoM}
\subsection{Linux}
\subsection{Windows}

\section{Конфігурація}

\section{Часті запитання та відповіді}

\section{Постскриптум}
Маємо дякувати за прочитання

\end{document}
